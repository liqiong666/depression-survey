% !TEX root = tnnls_relation_gait.tex

\ifx\allfiles\undefined
    % !TEX root = tnnls_relation_gait.tex

%% bare_jrnl.tex
%% V1.4b
%% 2015/08/26
%% by Michael Shell
%% see http://www.michaelshell.org/
%% for current contact information.
%%
%% This is a skeleton file demonstrating the use of IEEEtran.cls
%% (requires IEEEtran.cls version 1.8b or later) with an IEEE
%% journal paper.
%%
%% Support sites:
%% http://www.michaelshell.org/tex/ieeetran/
%% http://www.ctan.org/pkg/ieeetran
%% and
%% http://www.ieee.org/

%%*************************************************************************
%% Legal Notice:
%% This code is offered as-is without any warranty either expressed or
%% implied; without even the implied warranty of MERCHANTABILITY or
%% FITNESS FOR A PARTICULAR PURPOSE!
%% User assumes all risk.
%% In no event shall the IEEE or any contributor to this code be liable for
%% any damages or losses, including, but not limited to, incidental,
%% consequential, or any other damages, resulting from the use or misuse
%% of any information contained here.
%%
%% All comments are the opinions of their respective authors and are not
%% necessarily endorsed by the IEEE.
%%
%% This work is distributed under the LaTeX Project Public License (LPPL)
%% ( http://www.latex-project.org/ ) version 1.3, and may be freely used,
%% distributed and modified. A copy of the LPPL, version 1.3, is included
%% in the base LaTeX documentation of all distributions of LaTeX released
%% 2003/12/01 or later.
%% Retain all contribution notices and credits.
%% ** Modified files should be clearly indicated as such, including  **
%% ** renaming them and changing author support contact information. **
%%*************************************************************************


% *** Authors should verify (and, if needed, correct) their LaTeX system  ***
% *** with the testflow diagnostic prior to trusting their LaTeX platform ***
% *** with production work. The IEEE's font choices and paper sizes can   ***
% *** trigger bugs that do not appear when using other class files.       ***                          ***
% The testflow support page is at:
% http://www.michaelshell.org/tex/testflow/

\documentclass[journal]{IEEEtran}
%
% If IEEEtran.cls has not been installed into the LaTeX system files,
% manually specify the path to it like:
% \documentclass[journal]{../sty/IEEEtran}

% Some very useful LaTeX packages include:
% (uncomment the ones you want to load)

% *** MISC UTILITY PACKAGES ***
%
%\usepackage{ifpdf}
% Heiko Oberdiek's ifpdf.sty is very useful if you need conditional
% compilation based on whether the output is pdf or dvi.
% usage:
% \ifpdf
%   % pdf code
% \else
%   % dvi code
% \fi
% The latest version of ifpdf.sty can be obtained from:
% http://www.ctan.org/pkg/ifpdf
% Also, note that IEEEtran.cls V1.7 and later provides a builtin
% \ifCLASSINFOpdf conditional that works the same way.
% When switching from latex to pdflatex and vice-versa, the compiler may
% have to be run twice to clear warning/error messages.

% *** CITATION PACKAGES ***

\usepackage{tabularx}
\usepackage{longtable}
\usepackage{threeparttable}
\usepackage{cite}

% cite.sty was written by Donald Arseneau
% V1.6 and later of IEEEtran pre-defines the format of the cite.sty package
% \cite{} output to follow that of the IEEE. Loading the cite package will
% result in citation numbers being automatically sorted and properly
% "compressed/ranged". e.g., [1], [9], [2], [7], [5], [6] without using
% cite.sty will become [1], [2], [5]--[7], [9] using cite.sty. cite.sty's
% \cite will automatically add leading space, if needed. Use cite.sty's
% noadjust option (cite.sty V3.8 and later) if you want to turn this off
% such as if a citation ever needs to be enclosed in parenthesis.
% cite.sty is already installed on most LaTeX systems. Be sure and use
% version 5.0 (2009-03-20) and later if using hyperref.sty.
% The latest version can be obtained at:
% http://www.ctan.org/pkg/cite
% The documentation is contained in the cite.sty file itself.

% *** GRAPHICS RELATED PACKAGES ***
%
\usepackage[pdftex]{graphicx}
\usepackage{rotating}
\ifCLASSINFOpdf
  % \usepackage[pdftex]{graphicx}
  % declare the path(s) where your graphic files are
  % \graphicspath{{../pdf/}{../jpeg/}}
  % and their extensions so you won't have to specify these with
  % every instance of \includegraphics
  % \DeclareGraphicsExtensions{.pdf,.jpeg,.png}
\else
  % or other class option (dvipsone, dvipdf, if not using dvips). graphicx
  % will default to the driver specified in the system graphics.cfg if no
  % driver is specified.
  % \usepackage[dvips]{graphicx}
  % declare the path(s) where your graphic files are
  % \graphicspath{{../eps/}}
  % and their extensions so you won't have to specify these with
  % every instance of \includegraphics
  % \DeclareGraphicsExtensions{.eps}
\fi
% graphicx was written by David Carlisle and Sebastian Rahtz. It is
% required if you want graphics, photos, etc. graphicx.sty is already
% installed on most LaTeX systems. The latest version and documentation
% can be obtained at:
% http://www.ctan.org/pkg/graphicx
% Another good source of documentation is "Using Imported Graphics in
% LaTeX2e" by Keith Reckdahl which can be found at:
% http://www.ctan.org/pkg/epslatex
%
% latex, and pdflatex in dvi mode, support graphics in encapsulated
% postscript (.eps) format. pdflatex in pdf mode supports graphics
% in .pdf, .jpeg, .png and .mps (metapost) formats. Users should ensure
% that all non-photo figures use a vector format (.eps, .pdf, .mps) and
% not a bitmapped formats (.jpeg, .png). The IEEE frowns on bitmapped formats
% which can result in "jaggedy"/blurry rendering of lines and letters as
% well as large increases in file sizes.
%
% You can find documentation about the pdfTeX application at:
% http://www.tug.org/applications/pdftex

% *** MATH PACKAGES ***
%
\usepackage{amsmath}
% A popular package from the American Mathematical Society that provides
% many useful and powerful commands for dealing with mathematics.
%
% Note that the amsmath package sets \interdisplaylinepenalty to 10000
% thus preventing page breaks from occurring within multiline equations. Use:
%\interdisplaylinepenalty=2500
% after loading amsmath to restore such page breaks as IEEEtran.cls normally
% does. amsmath.sty is already installed on most LaTeX systems. The latest
% version and documentation can be obtained at:
% http://www.ctan.org/pkg/amsmath

% *** SPECIALIZED LIST PACKAGES ***
%
\usepackage{algorithmic}
% algorithmic.sty was written by Peter Williams and Rogerio Brito.
% This package provides an algorithmic environment fo describing algorithms.
% You can use the algorithmic environment in-text or within a figure
% environment to provide for a floating algorithm. Do NOT use the algorithm
% floating environment provided by algorithm.sty (by the same authors) or
% algorithm2e.sty (by Christophe Fiorio) as the IEEE does not use dedicated
% algorithm float types and packages that provide these will not provide
% correct IEEE style captions. The latest version and documentation of
% algorithmic.sty can be obtained at:
% http://www.ctan.org/pkg/algorithms
% Also of interest may be the (relatively newer and more customizable)
% algorithmicx.sty package by Szasz Janos:
% http://www.ctan.org/pkg/algorithmicx

% *** ALIGNMENT PACKAGES ***
%
\usepackage{array}
% Frank Mittelbach's and David Carlisle's array.sty patches and improves
% the standard LaTeX2e array and tabular environments to provide better
% appearance and additional user controls. As the default LaTeX2e table
% generation code is lacking to the point of almost being broken with
% respect to the quality of the end results, all users are strongly
% advised to use an enhanced (at the very least that provided by array.sty)
% set of table tools. array.sty is already installed on most systems. The
% latest version and documentation can be obtained at:
% http://www.ctan.org/pkg/array

% IEEEtran contains the IEEEeqnarray family of commands that can be used to
% generate multiline equations as well as matrices, tables, etc., of high
% quality.

% *** SUBFIGURE PACKAGES ***
\usepackage[caption=false,font=footnotesize]{subfig}
%\ifCLASSOPTIONcompsoc
%  \usepackage[caption=false,font=normalsize,labelfont=sf,textfont=sf]{subfig}
%\else
%  \usepackage[caption=false,font=footnotesize]{subfig}
%\fi
% subfig.sty, written by Steven Douglas Cochran, is the modern replacement
% for subfigure.sty, the latter of which is no longer maintained and is
% incompatible with some LaTeX packages including fixltx2e. However,
% subfig.sty requires and automatically loads Axel Sommerfeldt's caption.sty
% which will override IEEEtran.cls' handling of captions and this will result
% in non-IEEE style figure/table captions. To prevent this problem, be sure
% and invoke subfig.sty's "caption=false" package option (available since
% subfig.sty version 1.3, 2005/06/28) as this is will preserve IEEEtran.cls
% handling of captions.
% Note that the Computer Society format requires a larger sans serif font
% than the serif footnote size font used in traditional IEEE formatting
% and thus the need to invoke different subfig.sty package options depending
% on whether compsoc mode has been enabled.
%
% The latest version and documentation of subfig.sty can be obtained at:
% http://www.ctan.org/pkg/subfig

% *** FLOAT PACKAGES ***
%
%\usepackage{fixltx2e}
% fixltx2e, the successor to the earlier fix2col.sty, was written by
% Frank Mittelbach and David Carlisle. This package corrects a few problems
% in the LaTeX2e kernel, the most notable of which is that in current
% LaTeX2e releases, the ordering of single and double column floats is not
% guaranteed to be preserved. Thus, an unpatched LaTeX2e can allow a
% single column figure to be placed prior to an earlier double column
% figure.
% Be aware that LaTeX2e kernels dated 2015 and later have fixltx2e.sty's
% corrections already built into the system in which case a warning will
% be issued if an attempt is made to load fixltx2e.sty as it is no longer
% needed.
% The latest version and documentation can be found at:
% http://www.ctan.org/pkg/fixltx2e

%\usepackage{stfloats}
% stfloats.sty was written by Sigitas Tolusis. This package gives LaTeX2e
% the ability to do double column floats at the bottom of the page as well
% as the top. (e.g., "\begin{figure*}[!b]" is not normally possible in
% LaTeX2e). It also provides a command:
%\fnbelowfloat
% to enable the placement of footnotes below bottom floats (the standard
% LaTeX2e kernel puts them above bottom floats). This is an invasive package
% which rewrites many portions of the LaTeX2e float routines. It may not work
% with other packages that modify the LaTeX2e float routines. The latest
% version and documentation can be obtained at:
% http://www.ctan.org/pkg/stfloats
% Do not use the stfloats baselinefloat ability as the IEEE does not allow
% \baselineskip to stretch. Authors submitting work to the IEEE should note
% that the IEEE rarely uses double column equations and that authors should try
% to avoid such use. Do not be tempted to use the cuted.sty or midfloat.sty
% packages (also by Sigitas Tolusis) as the IEEE does not format its papers in
% such ways.
% Do not attempt to use stfloats with fixltx2e as they are incompatible.
% Instead, use Morten Hogholm'a dblfloatfix which combines the features
% of both fixltx2e and stfloats:
%
% \usepackage{dblfloatfix}
% The latest version can be found at:
% http://www.ctan.org/pkg/dblfloatfix

%\ifCLASSOPTIONcaptionsoff
%  \usepackage[nomarkers]{endfloat}
% \let\MYoriglatexcaption\caption
% \renewcommand{\caption}[2][\relax]{\MYoriglatexcaption[#2]{#2}}
%\fi
% endfloat.sty was written by James Darrell McCauley, Jeff Goldberg and
% Axel Sommerfeldt. This package may be useful when used in conjunction with
% IEEEtran.cls'  captionsoff option. Some IEEE journals/societies require that
% submissions have lists of figures/tables at the end of the paper and that
% figures/tables without any captions are placed on a page by themselves at
% the end of the document. If needed, the draftcls IEEEtran class option or
% \CLASSINPUTbaselinestretch interface can be used to increase the line
% spacing as well. Be sure and use the nomarkers option of endfloat to
% prevent endfloat from "marking" where the figures would have been placed
% in the text. The two hack lines of code above are a slight modification of
% that suggested by in the endfloat docs (section 8.4.1) to ensure that
% the full captions always appear in the list of figures/tables - even if
% the user used the short optional argument of \caption[]{}.
% IEEE papers do not typically make use of \caption[]'s optional argument,
% so this should not be an issue. A similar trick can be used to disable
% captions of packages such as subfig.sty that lack options to turn off
% the subcaptions:
% For subfig.sty:
% \let\MYorigsubfloat\subfloat
% \renewcommand{\subfloat}[2][\relax]{\MYorigsubfloat[]{#2}}
% However, the above trick will not work if both optional arguments of
% the \subfloat command are used. Furthermore, there needs to be a
% description of each subfigure *somewhere* and endfloat does not add
% subfigure captions to its list of figures. Thus, the best approach is to
% avoid the use of subfigure captions (many IEEE journals avoid them anyway)
% and instead reference/explain all the subfigures within the main caption.
% The latest version of endfloat.sty and its documentation can obtained at:
% http://www.ctan.org/pkg/endfloat
%
% The IEEEtran \ifCLASSOPTIONcaptionsoff conditional can also be used
% later in the document, say, to conditionally put the References on a
% page by themselves.

% *** PDF, URL AND HYPERLINK PACKAGES ***
%
\usepackage{url}
% url.sty was written by Donald Arseneau. It provides better support for
% handling and breaking URLs. url.sty is already installed on most LaTeX
% systems. The latest version and documentation can be obtained at:
% http://www.ctan.org/pkg/url
% Basically, \url{my_url_here}.

% *** Do not adjust lengths that control margins, column widths, etc. ***
% *** Do not use packages that alter fonts (such as pslatex).         ***
% There should be no need to do such things with IEEEtran.cls V1.6 and later.
% (Unless specifically asked to do so by the journal or conference you plan
% to submit to, of course. )

% correct bad hyphenation here
% \hyphenation{op-tical net-works semi-conduc-tor}
\usepackage{enumerate}
\usepackage{multirow}
\usepackage{color}
\usepackage{threeparttable}
\usepackage{booktabs}
\newcommand{\minus}{\scalebox{0.75}[1.0]{$-$}}
\newcommand{\bftab}[1]{{\fontseries{b}\selectfont#1}}
\newcommand{\tabincell}[2]{\begin{tabular}{@{}#1@{}}#2\end{tabular}}
\newcommand{\etal}{\textit{et al}.}
\newcommand{\ie}{\textit{i.e.}}
\newcommand{\eg}{\textit{e.g.}}
\newcommand{\wrt}{\textit{w.r.t.}}
\newcommand{\vs}{\textit{vs.}}


\begin{document}
%
% paper title
% Titles are generally capitalized except for words such as a, an, and, as,
% at, but, by, for, in, nor, of, on, or, the, to and up, which are usually
% not capitalized unless they are the first or last word of the title.
% Linebreaks \\ can be used within to get better formatting as desired.
% Do not put math or special symbols in the title.
%\title{The Application and Research of Multi-modal Data in Auxiliary Diagnosis of Depression: A Survey
%}
\title{ Auxiliary Diagnosis of Depression Based on Multi-modal Data: A Survey
}
%%
%%
%% author names and IEEE memberships
%% note positions of commas and nonbreaking spaces ( ~ ) LaTeX will not break
%% a structure at a ~ so this keeps an author's name from being broken across
%% two lines.
%% use \thanks{} to gain access to the first footnote area
%% a separate \thanks must be used for each paragraph as LaTeX2e's \thanks
%% was not built to handle multiple paragraphs
%%
%
%\author{Saihui~Hou,
%	Xu~Liu,
%	Chunshui~Cao,
%	and~Yongzhen~Huang$^*$% <-this % stops a space
%	\thanks{$^*$ indicates the corresponding author.}% <-this % stops a space
%    \thanks{Saihui Hou and Yongzhen Huang is with School of Artificial Intelligence, Beijing Normal University, Beijing 100875, China. (Email: housaihui@bnu.edu.cn, huangyongzhen@bnu.edu.cn)}
%	\thanks{Xu Liu and Chunshui Cao are with Watrix Technology Limited Co. Ltd, Beijing 100088, China. (Email: xu.liu@watrix.ai, chunshuicao@watrix.ai)}
%    \thanks{This work is partially supported by the Fundamental Research Funds for the Central Universities.}
%    % \thanks{E-mail: housaihui@bnu.edu.cn, xu.liu@watrix.ai, chunshuicao@watrix.ai, huangyongzhen@bnu.edu.cn}
%	% \thanks{Manuscript received April 19, 2005; revised August 26, 2015.}
%}
%
%% note the % following the last \IEEEmembership and also \thanks -
%% these prevent an unwanted space from occurring between the last author name
%% and the end of the author line. i.e., if you had this:
%%
%% \author{....lastname \thanks{...} \thanks{...} }
%%                     ^------------^------------^----Do not want these spaces!
%%
%% a space would be appended to the last name and could cause every name on that
%% line to be shifted left slightly. This is one of those "LaTeX things". For
%% instance, "\textbf{A} \textbf{B}" will typeset as "A B" not "AB". To get
%% "AB" then you have to do: "\textbf{A}\textbf{B}"
%% \thanks is no different in this regard, so shield the last } of each \thanks
%% that ends a line with a % and do not let a space in before the next \thanks.
%% Spaces after \IEEEmembership other than the last one are OK (and needed) as
%% you are supposed to have spaces between the names. For what it is worth,
%% this is a minor point as most people would not even notice if the said evil
%% space somehow managed to creep in.
%
%% The paper headers
%\markboth{IEEE Transactions on Neural Networks and Learning Systems}%
%{Saihui Hou \MakeLowercase{\textit{et al.}}: GQAN: Towards the Interpretability of Silhouette-based Gait Recognition}
%% The only time the second header will appear is for the odd numbered pages
%% after the title page when using the twoside option.
%%
%% *** Note that you probably will NOT want to include the author's ***
%% *** name in the headers of peer review papers.                   ***
%% You can use \ifCLASSOPTIONpeerreview for conditional compilation here if
%% you desire.
%
%% If you want to put a publisher's ID mark on the page you can do it like
%% this:
%%\IEEEpubid{0000--0000/00\$00.00~\copyright~2015 IEEE}
%% Remember, if you use this you must call \IEEEpubidadjcol in the second
%% column for its text to clear the IEEEpubid mark.
%
%% use for special paper notices
%%\IEEEspecialpapernotice{(Invited Paper)}
%
%% make the title area
\maketitle

\fi
%
%%%%%%%%%%%%%%%%%%%%%%%%%%%%%%%%%
%\begin{table*}[htbp]
%	\caption{
%		The rank-1 accuracy (\%) on CASIA-B for different probe views excluding the identical-view cases.
%		For evaluation, the sequences of NM-1,2,3,4 for each subject are taken as the gallery.
%		% The probe contains the sequences of three walking conditions, \ie~NM, BG and CL.
%	}
%	\label{tab_acc_casia}
%	\begin{center}
%		\resizebox{0.9999\textwidth}{!}{%
%			\begin{tabular}{c|c|c|c|c|c|c|c|c|c|c|c|c|c}
%				\hline
%				\multirow{2}{*}{\tabincell{c}{}} & \multirow{2}{*}{Method} & \multicolumn{11}{c|}{Probe View} & \multirow{2}{*}{Average} \\
%				\cline{3-13}
%				& & $0^{\circ}$ & $18^{\circ}$ & $36^{\circ}$ & $54^{\circ}$ & $72^{\circ}$ & $90^{\circ}$
%				& $108^{\circ}$ & $126^{\circ}$ & $134^{\circ}$ & $162^{\circ}$ &  $180^{\circ}$ & \\
%				\hline
%				\multirow{8}{*}{NM}
%				%%%%%%%%%%%%%%%%%%%%%%%%%%%%%%%%%%%%%%%
%				& GEINet~\cite{shiraga2016geinet}          & 40.20 & 38.90 & 42.90 & 45.60 & 51.20 & 42.00 & 53.50 & 57.60 & 57.80 & 51.80 & 47.70 & 48.11 \\
%				& CNN-LB~\cite{wu2016comprehensive}        & 82.60 & 90.30 & 96.10 & 94.30 & 90.10 & 87.40 & 89.90 & 94.00 & 94.70 & 91.30 & 78.50 & 89.93 \\
%				& GaitSet~\cite{chao2019gaitset}           & 93.40 & 98.10 & 98.50 & 97.80 & 92.60 & 90.90 & 94.20 & 97.30 & 98.40 & 97.00 & 89.10 & 95.21 \\
%				& GaitPart~\cite{fan2020gaitpart}          & 94.10 & 98.60 & 99.30 & 98.50 & 94.00 & 92.30 & 95.90 & 98.40 & 99.20 & 97.80 & 90.40 & 96.23 \\
%				& GLN~\cite{hou2020gait}                   & 93.20 & 99.30 & 99.50 & 98.70 & 96.10 & 95.60 & 97.20 & 98.10 & 99.30 & 98.60 & 90.10 & 96.88 \\
%				& SRN~\cite{hou2021setres}                 & 94.70 & 99.40 & 99.40 & 98.40 & 96.50 & 94.80 & 96.00 & 98.20 & 99.30 & 98.40 & 92.90 & 97.09 \\
%				& GQAN-Backbone(\bftab{ours})              & 95.80 & 99.60 & 99.70 & 99.10 & \bftab{97.90} & 96.30 & \bftab{97.80} & 98.70 & 99.50 & 98.30 & 94.10 & 97.89 \\
%				& GQAN(\bftab{ours})                       & \bftab{98.00} & \bftab{99.80} & \bftab{99.80} & \bftab{99.20} & 97.70 & \bftab{97.30} & \bftab{97.80} & \bftab{98.80} & \bftab{99.80} & \bftab{99.20} & \bftab{96.20} & \bftab{98.51} \\
%				%%%%%%%%%%%%%%%%%%%%%%%%%%%%%%%%%%%%%%%
%				\hline
%				\multirow{8}{*}{BG}
%				%%%%%%%%%%%%%%%%%%%%%%%%%%%%%%%%%%%%%%%
%				& GEINet~\cite{shiraga2016geinet}          & 34.20 & 29.29 & 31.21 & 35.20 & 35.20 & 27.60 & 35.90 & 43.50 & 45.00 & 38.99 & 36.80 & 35.72 \\
%				& CNN-LB~\cite{wu2016comprehensive}        & 64.20 & 80.60 & 82.70 & 76.90 & 64.80 & 63.10 & 68.00 & 76.90 & 82.20 & 75.40 & 61.30 & 72.37 \\
%				& GaitSet~\cite{chao2019gaitset}           & 85.90 & 92.12 & 93.94 & 90.41 & 86.40 & 78.70 & 85.00 & 91.60 & 93.10 & 91.01 & 80.70 & 88.08 \\
%				& GaitPart~\cite{fan2020gaitpart}          & 89.10 & 94.80 & 96.70 & 95.10 & 88.30 & 84.90 & 89.00 & 93.50 & 96.10 & 93.80 & 85.80 & 91.55 \\
%				& GLN~\cite{hou2020gait}                   & 91.10 & 97.68 & 97.78 & 95.20 & 92.50 & 91.20 & 92.40 & \bftab{96.00} & 97.50 & 94.95 & 88.10 & 94.04 \\
%				& SRN~\cite{hou2021setres}                 & 92.00 & 97.37 & 97.58 & 95.82 & 91.80 & 90.40 & 93.20 & 95.30 & 97.60 & 95.35 & 87.80 & 94.02 \\
%				& GQAN-Backbone(\bftab{ours})              & 93.90 & 97.27 & 97.37 & 96.43 & \bftab{94.00} & \bftab{92.60} & 93.10 & 95.40 & 97.40 & 96.97 & 88.70 & 94.83 \\
%				& GQAN(\bftab{ours})                       & \bftab{96.00} & \bftab{98.69} & \bftab{98.38} & \bftab{96.94} & 93.40 & 90.80 & \bftab{93.70} & 95.90 & \bftab{97.70} & \bftab{97.07} & \bftab{90.50} & \bftab{95.37} \\
%				%%%%%%%%%%%%%%%%%%%%%%%%%%%%%%%%%%%%%%%
%				\hline
%				\multirow{8}{*}{CL}
%				%%%%%%%%%%%%%%%%%%%%%%%%%%%%%%%%%%%%%%%
%				& GEINet~\cite{shiraga2016geinet}          & 19.90 & 20.30 & 22.50 & 23.50 & 26.70 & 21.30 & 27.40 & 28.20 & 24.20 & 22.50 & 21.60 & 23.46 \\
%				& CNN-LB~\cite{wu2016comprehensive}        & 37.70 & 57.20 & 66.60 & 61.10 & 55.20 & 54.60 & 55.20 & 59.10 & 58.90 & 48.80 & 39.40 & 53.98 \\
%				& GaitSet~\cite{chao2019gaitset}           & 63.70 & 75.60 & 80.70 & 77.50 & 69.10 & 67.80 & 69.70 & 74.60 & 76.10 & 71.10 & 55.70 & 71.05 \\
%				& GaitPart~\cite{fan2020gaitpart}          & 70.70 & 85.50 & 86.90 & 83.30 & 77.10 & 72.50 & 76.90 & 82.20 & 83.80 & 80.20 & 66.50 & 78.69 \\
%				& GLN~\cite{hou2020gait}                   & 70.60 & 82.40 & 85.20 & 82.70 & 79.20 & 76.40 & 76.20 & 78.90 & 77.90 & 78.70 & 64.30 & 77.50 \\
%				& SRN~\cite{hou2021setres}                 & 75.10 & 88.20 & 89.90 & 86.30 & 81.20 & 78.80 & 80.00 & 84.00 & 86.30 & 80.70 & 68.80 & 81.75 \\
%				& GQAN-Backbone(\bftab{ours})              & 71.70 & 84.00 & 88.70 & 84.30 & 83.20 & 78.30 & 81.80 & 83.20 & 83.60 & 77.50 & 66.10 & 80.22 \\
%				& GQAN(\bftab{ours})                       & \bftab{80.20} & \bftab{90.30} & \bftab{90.20} & \bftab{87.40} & \bftab{85.50} & \bftab{81.50} & \bftab{83.70} & \bftab{85.30} & \bftab{86.90} & \bftab{83.30} & \bftab{75.30} & \bftab{84.51} \\
%				%%%%%%%%%%%%%%%%%%%%%%%%%%%%%%%%%%%%%%%
%				\hline
%			\end{tabular}
%		}
%	\end{center}
%\end{table*}
%%%%%%%%%%%%%%%%%%%%%%%%%%%%%%%%%

\section{Experiments}

\subsection{Experimental Settings}
The experimental settings for GQAN are similar to the baseline methods~\cite{hou2020gait,hou2021setres,chao2019gaitset} to ensure the fair comparisons.
%
The methods in~\cite{hou2020gait,hou2021setres} are our previous works which aim to learn more discriminative features from the silhouettes for gait recognition.
%
Differently, the main goal of GQAN is to enhance the \emph{interpretability} of silhouette-based gait recognition, which also achieves very competitive performance under all walking conditions.

\subsubsection{GQAN-Backbone}
\label{sec_settings_backbone}
In our experiments, we design an effective and efficient backbone for GQAN.
%
Specifically, GQAN-Backbone is modified from GaitSet~\cite{chao2019gaitset} and our modifications mainly lie in the follows:
\begin{enumerate}[(i)]
	\item We use $S \! = \! 16$ instead of $S \!= \! \{1, 2, 4, 8, 16\}$ for simplicity to horizontally slice the features in Horizontal Pyramid Matching.
	\item We remove Multilayer Global Pipeline to make it feasible to separately assess the quality of each silhouette in the high layers, which can also accelerate the training and reduce the GPU memory consumption.
	\item We add the BNNeck~\cite{luo2019bag} and compute the cross-entropy loss on the concatenated features of all parts.
	\item We use the warmup strategy~\cite{he2016deep} to adjust the learning rate at the start of training.
	\item We adopt the random erasing data augmentation~\cite{zhong2020random} to alleviate the overfitting on CASIA-B~\cite{yu2006framework}.
	\item We add two additional convolutional layers in the Encoder for the experiments on OUMVLP~\cite{takemura2018multi} to adapt to the large-scale dataset.
\end{enumerate}
It is worth noting that, the networks in GaitSet~\cite{chao2019gaitset} as well as~\cite{fan2020gaitpart,hou2020gait,hou2021setres} cannot be directly adopted as the backbone for GQAN.
%
Specifically, the networks in GaitSet~\cite{chao2019gaitset}, GLN~\cite{hou2020gait} and SRN~\cite{hou2021setres} consist of a global branch (\eg, Multilayer Global Pipeline in GaitSet~\cite{chao2019gaitset}) to aggregate the silhouette-level features at the early layers which makes it infeasible to separately assess the quality of each silhouette in the high layers.
%
Besides, GaitPart~\cite{fan2020gaitpart} relies on a MCM module to model the micro-motion features in the adjacent frames and we provide the comparison between FQBlock and MCM in Section~\ref{sec_mcm_comparison}.

%%%%%%%%%%%%%%%%%%%%%%%%%%%%%%%%%%%%%%%%%%%%%%%%%%%%%%%%%%%%%%%%%%%%%%%%%%%%%%%%%%%%%%
\subsubsection{Datasets}
The experiments are mainly conducted on two typical gait datasets, \ie, CASIA-B~\cite{yu2006framework} and OUMVLP~\cite{takemura2018multi}.
%
% The dataset statistics are summarized in Table~\ref{tab_dataset}.

%%%%%%%%%%%%%%%%%%%%%%%%%%%%%%%%%%%%%%%%%%%
CASIA-B consists of $124$ subjects and collects the videos of normal walking (NM-1,2,3,4,5,6), walking with bags (BG-1,2) and walking in different coats/jackets (CL-1,2).
%
There are $11$ views for each walking condition.
%
Since there is not split way provided in the dataset, we take the first $74$ subjects as training set with the rest $50$ subjects as test set.
%
For evaluation, the sequences of NM-1,2,3,4 for each subject are taken as the gallery and the sequences of NM-5,6, BG-1,2, CL-1,2 are taken as the probe.

%%%%%%%%%%%%%%%%%%%%%%%%%%%%%%%%%%%%%%%%%%%
OUMVLP consists of $10307$ subjects which is the largest public gait dataset so far.
%
However, it only provides the silhouettes of normal walking (NM-00,01) for each subject.
%
There are $14$ views available for normal walking.
%
According to the split way provided in the dataset, we take the $5153$ subjects as training set with the rest $5154$ subjects as test set.
%
For evaluation, the sequences of NM-01 for each subject are taken as the gallery and the sequences of NM-00 are taken as the probe.

%%%%%%%%%%%%%%%%%%%%%%%%%%%%%%%%%%%%%%%%%%%%%%%%%%%%%%%%%%%%%%%%%%%%%%%%%%%%%%%%%%%%%%
\subsubsection{Implementation Details}
All models are implemented with PyTorch~\cite{NeurIPS2019_9015} and trained on TITAN-V GPUs.

%%%%%%%%%%%%%%%%%%%%%%%%%%%%%%%%%%%%%%%%%%%
The silhouettes in both datasets are pre-processed using the method in~\cite{takemura2017input}.
%
The input size of each silhouette is set to $128 \times 88$ for CASIA-B and $64 \times 44$ for OUMVLP.
%
In the training phase, we randomly select $30$ silhouettes from each sequence as the input.
%
The number of the subjects and the number of the sequences for each subject in a batch are set to $(8, 16)$ for CASIA-B and $(32, 16)$ for OUMVLP.
%
For evaluation, all silhouettes for each sequence are taken as the input to obtain the representations.

%%%%%%%%%%%%%%%%%%%%%%%%%%%%%%%%%%%%%%%%%%%%%%%%%%%%%%
The convolutional channels are set to $\{32, 64, 128\}$ for CASIA-B and $\{64, 128, 256, 512\}$ for OUMVLP.
%
The output dimension for each part representation is set to $256$.
%
For FQBlock, there is no dimension reduction and the channels in the two fully connected layers are the same as the output channel of the last convolutional layer ($128$ for CASIA-B and $512$ for OUMVLP).
%
For PQBlock, the output dimension of the fully connected layer is set to $1$.

%%%%%%%%%%%%%%%%%%%%%%%%%%%%%%%%%%%%%%%%%%%%%%%%%%%%%%
SGD with momentum is taken as the optimizer.
%
The learning rate is initialized with $0.1$ and scaled to its $1/10$ three times for the training.
%
The stepsize is set to $10000$ iterations for CASIA-B and $50000$ iterations for OUMVLP.
%
The momentum and the weight decay are set to $0.9$ and $0.0005$ for the optimization.
%
Particularly, GQAN is pretrained without PQLoss using the initial learning rate $0.1$ for $10000$ iterations for CASIA-B and $50000$ iterations for OUMVLP.

%%%%%%%%%%%%%%%%%%%%%%%%%%%%%%%%%%%%%%%%%%%%%%%%%%%%%%
Besides, the margin thresholds $m_{ap}$ and $m_{an}$ in Eq~\eqref{eq_pquality_lpq} are both set to $0.01$,
%
the margin threshold $m$ in Eq~\eqref{eq_summary_tp} is set to $0.2$,
%
the loss weight $\alpha$ in Eq~\eqref{eq_summary_l1} and Eq~\eqref{eq_summary_l2} is set $0.1$,
%
the loss weight $\beta$ in Eq~\eqref{eq_summary_l2} is set to $10.0$.


%%%%%%%%%%%%%%%%%%%%%%%%%%%%%%%%
\begin{table*}[!tbp]
	\caption{
		The rank-1 accuracy (\%) on OUMVLP for different probe views excluding the identical-view cases.
		For evaluation, the sequences of NM-01 for each subject are taken as the gallery.
        The probe sequences which have no corresponding ones in the gallery are included and ignored respectively.
		% The last three rows show the results ignoring the probe sequences which have no corresponding ones in the gallery.
	}
	\label{tab_acc_oumvlp}
	\begin{center}
		\resizebox{0.9999\textwidth}{!}{%
			\begin{tabular}{c|c|c|c|c|c|c|c|c|c|c|c|c|c|c|c}
				\hline
				\multirow{2}{*}{Method} & \multicolumn{14}{c|}{Probe View} & \multirow{2}{*}{Average} \\
				\cline{2-15}
				& $0^{\circ}$ & $15^{\circ}$ & $30^{\circ}$ & $45^{\circ}$ & $60^{\circ}$ & $75^{\circ}$ & $90^{\circ}$
				& $180^{\circ}$ & $195^{\circ}$ & $210^{\circ}$ & $225^{\circ}$ &  $240^{\circ}$ &  $255^{\circ}$ &  $270^{\circ}$ & \\
				\hline
				%%%%%%%%%%%%%%%%%%%%%%%%%%%%%%%%%%%%%%%
				GEINet~\cite{shiraga2016geinet}          & 23.20 & 38.09 & 47.95 & 51.81 & 47.53 & 48.09 & 43.75 & 27.25 & 37.89 & 46.78 & 49.85 & 45.94 & 45.65 & 40.96 & 42.48 \\
				GaitSet~\cite{chao2019gaitset}           & 79.33 & 87.59 & 89.96 & 90.09 & 87.96 & 88.74 & 87.69 & 81.82 & 86.46 & 88.95 & 89.17 & 87.16 & 87.60 & 86.15 & 87.05 \\
				GaitPart~\cite{fan2020gaitpart}          & 82.57 & 88.93 & 90.84 & 91.00 & 89.75 & 89.91 & 89.50 & 85.19 & 88.09 & 90.02 & 90.15 & 89.03 & 89.10 & 88.24 & 88.74 \\
				GLN~\cite{hou2020gait}                   & 83.81 & 90.00 & 91.02 & 91.21 & 90.25 & 89.99 & 89.43 & 85.28 & 89.09 & \bftab{90.47} & 90.59 & 89.60 & 89.31 & 88.47 & 89.18 \\
				SRN~\cite{hou2021setres}                 & 83.76 & 89.70 & 90.94 & 91.19 & 89.88 & 90.25 & 89.61 & 85.76 & 88.79 & 90.11 & 90.41 & 89.03 & 89.36 & 88.47 & 89.09 \\
				GQAN-Backbone(\bftab{ours})              & 84.38 & 90.06 & 91.15 & 91.30 & 90.41 & 90.41 & 89.86 & 86.85 & 89.13 & 90.34 & 90.51 & 89.75 & 89.50 & 88.77 & 89.46 \\
				GQAN(\bftab{ours})                       & \bftab{84.99} & \bftab{90.34} & \bftab{91.26} & \bftab{91.40} & \bftab{90.63} & \bftab{90.57} & \bftab{90.14} & \bftab{87.09} & \bftab{89.37} & 90.46 & \bftab{90.64} & \bftab{90.02} & \bftab{89.81} & \bftab{89.10} & \bftab{89.70} \\
				%%%%%%%%%%%%%%%%%%%%%%%%%%%%%%%%%%%%%%%
				\hline
				%%%%%%%%%%%%%%%%%%%%%%%%%%%%%%%%%%%%%%%
				GEINet~\cite{shiraga2016geinet}          & 24.91 & 40.65 & 51.55 & 55.13 & 49.81 & 51.05 & 46.37 & 29.17 & 40.67 & 50.53 & 53.27 & 48.39 & 48.64 & 43.49 & 45.26 \\
				GaitSet~\cite{chao2019gaitset}           & 84.50 & 93.27 & 96.72 & 96.58 & 93.48 & 95.28 & 94.15 & 87.04 & 92.50 & 96.00 & 95.96 & 92.99 & 94.34 & 92.69 & 93.25 \\
				GaitPart~\cite{fan2020gaitpart}          & 87.95 & 94.70 & 97.69 & 97.59 & 95.46 & 96.60 & 96.15 & 90.61 & 94.25 & 97.17 & 97.06 & 95.07 & 96.02 & 95.02 & 95.10 \\
				GLN~\cite{hou2020gait}                   & 89.28 & 95.84 & 97.87 & 97.82 & 96.01 & 96.68 & 96.07 & 90.71 & 95.34 & \bftab{97.66} & 97.54 & 95.69 & 96.24 & 95.27 & 95.57 \\
				SRN~\cite{hou2021setres}                 & 89.22 & 95.52 & 97.79 & 97.81 & 95.62 & 96.97 & 96.28 & 91.22 & 95.01 & 97.26 & 97.35 & 95.07 & 96.31 & 95.28 & 95.48 \\
				GQAN-Backbone(\bftab{ours})              & 89.88 & 95.92 & 98.03 & 97.94 & 96.20 & 97.17 & 96.57 & 92.37 & 95.37 & 97.51 & 97.46 & 95.88 & 96.48 & 95.62 & 95.89 \\
				GQAN(\bftab{ours})                       & \bftab{90.53} & \bftab{96.20} & \bftab{98.14} & \bftab{98.04} & \bftab{96.44} & \bftab{97.34} & \bftab{96.88} & \bftab{92.63} & \bftab{95.63} & 97.64 & \bftab{97.61} & \bftab{96.18} & \bftab{96.82} & \bftab{95.98} & \bftab{96.15} \\
				%%%%%%%%%%%%%%%%%%%%%%%%%%%%%%%%
				\hline
				%%%%%%%%%%%%%%%%%%%%%%%%%%%%%%%%%%%%%%%
			\end{tabular}
		}
	\end{center}
\end{table*}
%%%%%%%%%%%%%%%%%%%%%%%%%%%%%%%%

%%%%%%%%%%%%%%%%%%%%%%%%%%%%%%%%
\subsection{Performance Comparison}

\subsubsection{CASIA-B}
Table~\ref{tab_acc_casia} shows the performance comparison on CASIA-B.
%
The probe sequences are divided into three categories according to the walking conditions, \ie, NM, BG, CL, which are respectively evaluated.
%
We report the rank-1 accuracy for each probe view averaged on all gallery views excluding the identical-view cases~\cite{wu2016comprehensive}.

In the methods listed in Table~\ref{tab_acc_casia}, GEINet~\cite{shiraga2016geinet} and CNN-LB~\cite{wu2016comprehensive} are two representative methods taking GEIs as the input.
%
GaitSet~\cite{chao2019gaitset} first proposes to treat the silhouettes of a gait sequence as an unordered set and horizontally slices the features to learn part representation for gait recognition, which achieves significant improvement compared to the GEI-based methods.
%
GaitPart~\cite{fan2020gaitpart} uses a Focal Convolutional Layer and Micro-Motion Capture Module to enhance the part representations.
%
GLN~\cite{hou2020gait} takes the lateral connections to merge multi-layer features and proposes a Compact Block to reduce the representation dimension.
%
% For a comprehensive study, we also provide the comparison with the backbone of GLN without Compact Block (denoted as GLN-Backbone).
SRN~\cite{hou2021setres} proposes a Set Residual Block to effectively coordinate the silhouette-level and set-level information in the feature learning.

%%%%%%%%%%%%%%%%%%%%%%%%%%%%%%%%%%%%%%%%%%%%%%%%%%%%%%
For the results in Table~\ref{tab_acc_casia}, we can observe that the backbone we design for GQAN achieves the competitive performance (NM-$97.89\%$, BG-$94.83\%$, CL-$80.22\%$) compared to the previous works.
%
FQBlock and PQBlock for GQAN, which are proposed to assess the quality of each silhouette and each part for gait recognition, can further boost the performance under all walking conditions to state-of-the-art (NM-$98.51\%$, BG-$95.37\%$, CL-$84.51\%$).
%
Particularly, under the most challenging condition of walking in different coats/jackets, the rank-1 accuracy achieved by GQAN exceeds GQAN-Backbone by a large margin ($+4.29\%$).
%
In Section~\ref{sec_settings_backbone}, we have explained why the networks in~\cite{chao2019gaitset,fan2020gaitpart,hou2020gait,hou2021setres} cannot be directly adopted as the backbone for GQAN.
%
While in Table~\ref{tab_acc_casia}, GQAN-Backbone and GQAN constitute a fair comparison, and the performance gain brought by GQAN validates the effectiveness of FQBlock and PQBlock.

%%%%%%%%%%%%%%%%%%%%%%%%%%%%%%%%
\begin{table}[!tbp]
	\caption{
	The performance comparison on HID Competition Dataset 2021. The results are reported in the rank-1 accuracy.
	}
	\label{tab_hid}
	\begin{center}
		\resizebox{0.9999\linewidth}{!}{%
			\begin{tabular}{c|c|c|c}
				\hline
				Method & ~~~~SRN~\cite{hou2021setres}~~~~ & GQAN-Backbone & ~~~~~GQAN~~~~~ \\
				\hline
				Rank-1 Acc & 64.31 & 58.19 & \bftab{65.61} \\
				\hline
			\end{tabular}
		}
	\end{center}
\end{table}
%%%%%%%%%%%%%%%%%%%%%%%%%%%%%%%%

%%%%%%%%%%%%%%%%%%%%%%%%%%%%%%%%%%%%%%%%%%%%%%%%%%%%%%
\subsubsection{OUMVLP}
Table~\ref{tab_acc_oumvlp} shows the performance comparison on OUMVLP.
%
Though a large number of subjects are available, the lack of walking with bags (BG) and walking in different clothes (CL) makes it less challenging than CASIA-B.
%
Due to the incomplete data for some subjects, we respectively conduct the evaluation \emph{including} and \emph{ignoring} the probe sequences which have no corresponding ones in the gallery.

Compared to the methods listed in Table~\ref{tab_acc_casia}, CNN-LB~\cite{wu2016comprehensive} is too time-consuming for training and test which is thus not listed in Table~\ref{tab_acc_oumvlp} for the large-scale dataset.
%
% As far as we know, GLN~\cite{hou2020gait} holds the best performance (\ie, $95.57\%$) on OUMVLP before this work.
%
% In comparison, GQAN-Backbone achieves the rank-1 accuracy of $95.89\%$ and outperforms all the baselines, which validates the effectiveness of the backbone we designed for GQAN.
%
Here we mainly compare the rank-1 accuracy obtained by \emph{ignoring} the probe sequences which do not have the corresponding ones in the gallery.
%
From the results in Table~\ref{tab_acc_oumvlp}, we can observe that the baselines including GaitPart~\cite{fan2020gaitpart}, GLN~\cite{hou2020gait}, SRN~\cite{hou2021setres} all report the rank-1 accuracy of more than $95\%$ on this dataset.
%
Particularly, GQAN-Backbone achieves the competitive accuracy of $95.89\%$ which validates the effectiveness of the backbone we designed for GQAN.
%
Besides, we notice that the frame quality of each silhouette in OUMVLP is obviously higher than that in CASIA-B and the silhouettes for each subject walking in different clothes (CL) are not available.
%
FQBlock and PQBlock, which work by explicitly assessing the quality of each silhouette and each part, can still benefit the gait recognition and improve the rank-1 accuracy to $96.15\%$.

\subsubsection{HID Competition Dataset 2021}
CASIA-B and OUMVLP are the two most popular benchmarks which are widely used in previous literature~\cite{chao2019gaitset,fan2020gaitpart,hou2020gait,hou2021setres}.
%
The gait dataset is lacking due to the privacy issue, the requirement for cameras and sites.
%
To further verify the effectiveness of GQAN, we conduct the experiments on HID Competition Dataset 2021~\cite{yu2021hid} using the settings similar to CASIA-B.
%
The competition provides the sequences for 500 subjects and each subject consists of about 10 sequences.
%
We take the sequences of the first 300 subjects as training set and the rest 200 subjects as test set.
%
For evaluation, we gather the first sequence of each subject as the gallery and take the rest sequences as the probe.
%
The performance comparison is provided in Table~\ref{tab_hid} and the results are reported in the rank-1 accuracy.
%
Along with GQAN-Backbone and GQAN, we also re-implement SRN on this dataset which holds the best performance for gait recognition before this work.
%
The performance comparison shown in Table~\ref{tab_hid}, especially the performance gain from GQAN-Backbone to GQAN, further validates the effectiveness of the proposed method.

%%%%%%%%%%%%%%%%%%%%%%%%%%%%%%%%
\begin{table}[!tbp]
	\caption{
		The effect of each block for GQAN. The results are reported in the rank-1 accuracy excluding the identical-view cases.
		% \emph{GQAN-B} for \emph{GQAN-Backbone}.
	}
	\label{tab_ablation}
	\begin{center}
		\resizebox{0.9999\linewidth}{!}{%
			\begin{tabular}{c|ccc|c}
				\hline
				Dataset & \multicolumn{3}{c|}{CASIA-B} & OUMVLP \\
				\hline
				Method  			 		 & NM & BG & CL & NM \\
				\hline
				GQAN-Backbone        		 & 97.89 & 94.83 & 80.22 & 95.89 \\
				GQAN-Backbone+FQBlock 		 & 98.60 & 95.41 & 83.72 & 96.01 \\
				GQAN-Backbone+PQBlock		 & 97.96 & 94.68 & 82.36 & 96.01 \\
				GQAN           		         & 98.51 & 95.37 & 84.51 & 96.15 \\
				\hline
			\end{tabular}
		}
	\end{center}
\end{table}
%%%%%%%%%%%%%%%%%%%%%%%%%%%%%%%%

%%%%%%%%%%%%%%%%%%%%%%%%%%%%%%%%%%%%%%%%%%%%%%%%%%%
\begin{figure*}[tbp]
	\centering
	\subfloat[ID=110, Type=BG-01, View=036, Bin=1]{
		\label{fig_fquality1}\includegraphics[width=0.492\linewidth]{figures/png_id_110_type_bg-01_view_036_visualize_fquality_bin00.pdf}
	}
	\subfloat[ID=110, Type=BG-01, View=090, Bin=15]{
		\label{fig_fquality2}\includegraphics[width=0.492\linewidth]{figures/png_id_110_type_bg-01_view_090_visualize_fquality_bin14.pdf}
	}
	\\
	\vspace{-2mm}
	\subfloat[ID=112, Type=CL-02, View=018, Bin=1]{
		\label{fig_fquality3}\includegraphics[width=0.492\linewidth]{figures/png_id_112_type_cl-02_view_018_visualize_fquality_bin00.pdf}
	}
	\subfloat[ID=112, Type=NM-06, View=144, Bin=11]{
		\label{fig_fquality4}\includegraphics[width=0.492\linewidth]{figures/png_id_112_type_nm-06_view_144_visualize_fquality_bin10.pdf}
	}
	\\
	\vspace{-2mm}
	\subfloat[ID=123, Type=BG-02, View=018, Bin=4]{
		\label{fig_fquality5}\includegraphics[width=0.492\linewidth]{figures/png_id_123_type_bg-02_view_018_visualize_fquality_bin03.pdf}
	}
	\subfloat[ID=123, Type=CL-02, View=018, Bin=4]{
		\label{fig_fquality6}\includegraphics[width=0.492\linewidth]{figures/png_id_123_type_cl-02_view_018_visualize_fquality_bin03.pdf}
	}
	\caption{
		Examples for frame quality visualization on CASIA-B.
		%
		The silhouettes in each figure are randomly selected from each gait sequence and sorted in a descending order according to the frame quality scores of a certain bin.
		%
		The corresponding region in each silhouette is marked in red.
		%
		The number above each silhouette is the frame quality score which is computed by adding $Y_{ij}$ in Eq~\eqref{eq_fquality_yij} along the channel dimension.
		%
		% Best viewed in color.
	}
	\label{fig_fquality}
\end{figure*}
%%%%%%%%%%%%%%%%%%%%%%%%%%%%%%%%%%%%%%%%%%%%%%%%%%%

%%%%%%%%%%%%%%%%%%%%%%%%%%%%%%%%
\subsection{Ablation Study}
In this section, we conduct more experiments to further analyze GQAN.
% and the experiments are mainly conducted on CASIA-B covering different walking conditions.
%
For simplicity, we report the rank-1 accuracy under different walking conditions averaged on all probe and gallery views excluding the identical-view cases.

\subsubsection{The Effect of Each Block}
GQAN mainly consists of two blocks, \ie, FQBlock and PQBlock, to explicitly assess the quality of each silhouette and each part.
%
It is proposed towards the interpretability of silhouette-based gait recognition which also achieves very competitive performance.
%
Here we conduct the experiments to separately evaluate the effect of each block on the recognition accuracy.
%
The experimental results are provided in Table~\ref{tab_ablation}.
%
From the results on CASIA-B, we can observe that FQBlock can improve the performance for all walking conditions (NN, BG, CL).
%
The probable reason is that the silhouettes in CASIA-B are obtained by subtracting the background~\cite{yu2006framework} and contain a lot of noise.
%
PQBlock is mainly beneficial for walking in different coats/jackets (CL) which causes a lot of shape variance for the upper body.
%
For OUMVLP, as mentioned above, the overall improvement is not that significant due to the high quality of each silhouette and the lack of walking in different clothes (CL), while the performance comparison shown in Table~\ref{tab_ablation} indicates that FQBlock and PQBlock can still help improve the recognition accuracy respectively.
%
% the quality of each silhouette is relatively high and only the silhouettes of normal walking (NM) are available for each subject, while FQBlock and PQBlock can still help improve the recognition accuracy.

%%%%%%%%%%%%%%%%%%%%%%%%%%%%%%%%
\begin{table}[tbp]
	\caption{
		The performance comparison under the low-resolution conditions.
		The experiments are conducted on the test set of CASIA-B and the results are reported in the rank-1 accuracy excluding the identical-view cases.
	}
	\label{tab_diff_resolution}
	\begin{center}
		\resizebox{0.9999\linewidth}{!}{%
			\begin{tabular}{c|c|ccc}
				\hline
				Downsample Size & ~~~~~~~Method~~~~~~~ & NM & BG & CL \\
				\hline
				\multirow{2}{*}{-}
				& GQAN-Backbone & 97.89 & 94.83 & 80.22 \\
				& GQAN          & 98.51 & 95.37 & 84.51 \\
				\hline
				\multirow{2}{*}{ $64\times44$ }
				& GQAN-Backbone & 96.77 & 93.29 & 74.93 \\
				& GQAN          & 97.89 & 94.16 & 79.67 \\
				\hline
				\multirow{2}{*}{ $32\times22$ }
				& GQAN-Backbone & 72.11 & 65.11 & 37.29 \\
				& GQAN          & 76.63 & 69.85 & 44.10 \\
				\hline
			\end{tabular}
		}
	\end{center}
\end{table}
%%%%%%%%%%%%%%%%%%%%%%%%%%%%%%%%

\subsubsection{Performance Comparison under Low Resolution}
In this section, we provide the performance comparison between GQAN-Backbone and GQAN under the low-resolution conditions for a more comprehensive study.
%
Specifically, the experiments are conducted on the test set of CASIA-B covering different walking conditions.
%
In the above experiments, the input size of each silhouette is set to $128 \times 88$ for training and test.
%
While in this section, we first downsample the silhouettes in the test set and then restore to $128 \times 88$ to simulate the low-resolution conditions.
%
The results shown in Table~\ref{tab_diff_resolution} show that GQAN can consistently outperform the backbone, which indicates that the proposed method is more robust to the low-resolution conditions.

\subsubsection{Frame Quality Visualization}
\label{sec_frame_vis}
As described in Section~\ref{sec_fquality}, FQBlock works in a squeeze-and-excitation style to assess the frame quality of each silhouette for gait recognition.
%
Specifically, it predicts the scores $Y_{ij}$ as shown in Eq~\eqref{eq_fquality_yij} to recalibrate the features of the $i$-th silhouette and $j$-bin,
%
and we add $Y_{ij}$ along the channel dimension as the frame quality indicator for the $i$-th silhouette and $j$-th region.
%
In Fig.~\ref{fig_fquality}, we present some examples from CASIA-B and the silhouettes are sorted in descending order according to the predicted scores of FQBlock.
%
As aforementioned, the silhouettes in CASIA-B contain a lot of noise due to the errors in the background extraction~\cite{yu2006framework}.
%
From the results in Fig.~\ref{fig_fquality}, we can observe that $Y_{ij}$ can be treated as an indicator of the frame quality for each silhouette in a gait sequence.
%
% As mentioned in Section~\ref{sec_fquality}, FQBlock holds independent weights for different bins obtained by horizontally and equally slicing the features, and thus the orders of the silhouettes are different according to the $Y_{ij}$ with different subscript $j$.

%%%%%%%%%%%%%%%%%%%%%%%%%%%%%%%%%%%%%%%%%%%%%%%%%%%
\begin{figure}[t]
	\centering
	\includegraphics[width=1.0\linewidth]{figures/casia_b_visualize_pquality_cross.pdf}
	\caption{
		The statistics of part quality scores on CASIA-B.
		We conduct the experiment on the test set and average the scores for each part when computing the distance of gait sequences belonging to different types.
		\emph{REFER} for \emph{the reference scores of treating all parts equally (0.0625).}
	}
	\label{fig_pquality_cross}
\end{figure}
%%%%%%%%%%%%%%%%%%%%%%%%%%%%%%%%%%%%%%%%%%%%%%%%%%%

\subsubsection{Part Quality Visualization}
As described in Section~\ref{sec_pquality}, PQBlock operates on the set-level part representations and it predicts a score encoding the part quality to generate the adaptive weights as shown in Eq~\eqref{eq_pquality_ada}.
%
For the visualization of PQBlock, we obtain the statistics of part quality scores on the test set of CASIA-B when computing the distance of gait sequences belonging to different walking conditions, including NM-NM, NM-BG, NM-CL.
%
The results are displayed in Fig.~\ref{fig_pquality_cross}.
%
It can be observed that the part quality scores for the upper body ($6$-th to $10$-th part) are relatively low especially for the cases of NM-CL.
%
% The main reason is that the subjects of CASIA-B only change coats or jackets while the pants or skirts are the same for different walking conditions.
%
Besides, the mean score of $15$-th part is relatively smaller than the reference score of treating all parts equally, which is possibly caused by the segmentation errors due to the shadow in the floor~\cite{yu2006framework}.
%
In Fig.~\ref{fig_shadow_errors}, we display some example silhouettes from CASIA-B to illustrate the segmentation errors caused by the shadow in the floor.

%%%%%%%%%%%%%%%%%%%%%%%%%%%%%%%%%%%%%%%%%%%%%%%%%%%
\begin{figure}[t]
	\centering
	\includegraphics[width=1.0\linewidth]{figures/casia_b_visualize_shadow_errors_bin14.pdf}
	\caption{
		Example silhouettes from CASIA-B.
		The silhouettes in CASIA-B contain a lot of noise for the $15$-th region due to the shadow in the floor.
		The corresponding region in each silhouette is marked in red.
		% Best viewed in color.
	}
	\label{fig_shadow_errors}
\end{figure}
%%%%%%%%%%%%%%%%%%%%%%%%%%%%%%%%%%%%%%%%%%%%%%%%%%%

%%%%%%%%%%%%%%%%%%%%%%%%%%%%%%%%
\subsection{Discussion}
\label{sec_discussion}
\subsubsection{Comparison with MCM}
\label{sec_mcm_comparison}
In this section, we provide the comparison between FQBlock and MCM~\cite{fan2020gaitpart} which both take the part features extracted by the backbone as the input.
%
Firstly, the motivation behind the two blocks is different. MCM is designed to model the micro-motion features in the adjacent frames while FQBlock is proposed to measure the quality of each frame.
%
Secondly, the working mechanism of the two blocks is different. For simplicity, given a gait sequence, we denote the three dimensions of tensor to FQBlock and MCM as \emph{D-S}, \emph{D-P}, \emph{D-C}. Specifically, \emph{D-S} denotes the number of silhouettes, \emph{D-P} denotes the number of human parts, \emph{D-C} denotes the number of feature channels. MCM mainly works on \emph{D-S} dimension to deal with the relation of the adjacent frames, while FQBlock works on \emph{D-C} dimension to deal with the features of each frame separately.
%
Thirdly, the attention values in the two blocks have different characteristics. Specifically, the attention values of each frame in MCM rely on the adjacent ones, which are susceptible to temporal kernel size, silhouette order and missing frames. Moreover, the attention values for the frames in different sequences are not comparable due to the variation of context. As a result, the attention values in MCM are not suitable as the frame quality indicator. In contrast, the attention values of each frame in FQBlock only rely on the features of itself and are permutation invariant to silhouette order. Besides, FQBlock shares the weights across different silhouettes, which makes the attention values of the frames in different sequences comparable. The visualization results shown in Fig.~\ref{fig_fquality} indicate that the attention values in FQBlock can be taken as the frame quality indicator.

In addition, we conduct the experiment with MCM using the proposed strong baseline on CASIA-B. The performance of GQAN-Backbone+MCM (NM-97.96\%, BG-94.92\%, CL-81.74\%) is inferior to GQAN-Backbone+FQBlock (NM-98.60\%, BG-95.41\%, CL-83.72\%), which further demonstrates the effectiveness of FQBlock.

\subsubsection{Comparison with More Methods}
For a comprehensive study, we provide the performance comparison with more methods on CASIA-B and OUMVLP in Table~\ref{tab_more_baselines}.
%
Most of these works are orthogonal to GQAN such as the model-based methods (\eg, PoseGait~\cite{liao2020model} and End2EndGait~\cite{li2020end}) and the appearance methods taking other types of input for gait recognition (\eg, GaitNet~\cite{zhang2019gait} and GaitMotion~\cite{bashir2009gait}).
%
Particularly, SM-Prod~\cite{castro2020multimodal} reporting a little higher accuracy for NM and BG on CASIA-B. However, the optical flow needs a lot of computation cost and the performance for the most challenging CL is much inferior to GQAN.
%
Besides, SRN+CBlock~\cite{hou2021setres} is a variant of SRN which integrates SRN with Compact Block proposed in~\cite{hou2020gait}.
%
It reports a little higher accuracy on OUMVLP while its performance on CASIA-B is inferior to GQAN especially for the challenging CL ($77.7\%$~\vs~$84.51\%$).

%%%%%%%%%%%%%%%%%%%%%%%%%%%%%%%%
\begin{table}[tbp]
	\caption{
		The performance comparison with more baselines.
		The results are reported in the rank-1 accuracy excluding the identical-view cases.
		Sil for Silhouettes, RGB for RGB Frames, OF for Optical Flow.
	}
	\label{tab_more_baselines}
	\begin{center}
		\resizebox{0.9999\linewidth}{!}{%
			\begin{tabular}{c|c|c|ccc}
				\hline
				Dataset & Method & Input & NM & BG & CL \\
				\hline
				\multirow{11}{*}{CASIA-B}
				& J-CNN~\cite{zhang2019comprehensive}     & Sil & 91.2 & 75.0 & 54.0 \\
				& GaitSet-L~\cite{hou2020gait}   		  & Sil & 95.6 & 91.5 & 75.3 \\
				& GLN-Backbone~\cite{hou2020gait} 		  & Sil & 95.5 & 92.0 & 77.2 \\
				& SRN+CBlock~\cite{hou2021setres}         & Sil & 97.5 & 94.3 & 77.7 \\
				& MT3D~\cite{lin2020gait}	              & Sil & 96.7 & 93.0 & 81.5 \\
				& PoseGait~\cite{liao2020model}           & RGB & 68.7 & 44.5 & 36.0 \\
				& GaitNet~\cite{zhang2019gait}            & RGB & 92.3 & 88.9 & 62.3 \\
				& End2EndGait~\cite{li2020end} 		      & RGB & 97.9 & 93.1 & 77.6 \\
				& GaitMotion~\cite{bashir2009gait}        & OF  & 97.5 & 83.6 & 48.8 \\
				& SM-Prod~\cite{castro2020multimodal}     & Gray+OF	& \bftab{99.8} & \bftab{96.1} & 67.0 \\
				& GQAN(\bftab{ours}) 			          & Sil & 98.51 & 95.37 & \bftab{84.51} \\
				\hline
				\multirow{4}{*}{OUMVLP}
				& GLN-Backbone~\cite{hou2020gait} 		  & Sil  & 94.2 & - & - \\
				& SRN+CBlock~\cite{hou2021setres}         & Sil  & \bftab{96.4} & - & - \\
				& End2EndGait~\cite{li2020end} 		  	  & RGB  & 95.8 & - & - \\
				& GQAN(\bftab{ours}) 				      & Sil  & 96.15 & - & - \\
				\hline
			\end{tabular}
		}
	\end{center}
\end{table}
%%%%%%%%%%%%%%%%%%%%%%%%%%%%%%%%

\subsubsection{Discussion on Training Tricks}
As described in Section~\ref{sec_settings_backbone}, we adopt some useful tricks to train GQAN-Backbone and achieve the competitive performance.
%
In this section, we try to add these tricks to SRN~\cite{hou2021setres} which holds the best performance before this work.
%
We conduct the experiments on CASIA-B and the performance under different walking conditions is moderately improved (NM-98.01\%, BG-95.09\%, CL-83.34\%) which yet is still inferior to GQAN (NM-98.51\%, BG-95.37\%, CL-84.51\%).
%
More importantly, GQAN can enhance the interpretability of silhouette-based gait recognition by trying to find out the relative importance of each silhouette and each part.
%
% In Section~\ref{sec_settings_backbone}, we have explained the reasons why SRN~\cite{hou2021setres} as well as GaitSet~\cite{chao2019gaitset}, GaitPart~\cite{fan2020gaitpart}, GLN~\cite{hou2020gait} cannot be directly adopted as the backbone for GQAN.
%
% We believe that GQAN can achieve higher accuracy under different walking conditions with the aid of a stronger backbone, which is left for the future work.

\subsubsection{Discussion on Generalization Ability}
In this section, we conduct the experiments to verify the generalization ability of the learned gait quality.
%
Specifically, we perform the frame quality visualization on HID Competition Dataset 2021 using the same settings as those in Section~\ref{sec_frame_vis} except that the model is only trained on CASIA-B.
%
We provide some visualization results randomly selected from HID Competition Dataset 2021 in Fig.~\ref{fig_fquality_hid}, which indicates that the learned gait quality can be generalized to an unseen dataset.

\subsubsection{Discussion on Gait Interpretability}
The interpretability of gait recognition is greatly important for the real-world applications.
%
In this work, we move towards the interpretability of silhouette-based gait recognition by explicitly assessing the quality of each silhouette and each part.
%
However, the problem needs further exploration.
%
For example, the part representations for gait recognition in most works~\cite{chao2019gaitset,fan2020gaitpart,hou2020gait} are obtained by horizontally and equally slicing the features which are inconsistent with the semantic parts of human body, \eg, hands or feet.
%
And the interpretability of model-based gait recognition also needs to be explored where the weights for different key points are not explicitly modeled in the previous works~\cite{liao2020model,li2020end}.
%
Moreover, fusing the multimodal features proves to useful for many other visual tasks~\cite{zhang2020advances,zhang2021improved}.
%
It is also promising to fuse the silhouettes and key points to obtain more rich features for gait recognition, and how to enhance the interpretability in this case remains a challenge.

\ifx\allfiles\undefined
% !TEX root = tnnls_relation_gait.tex

% if have a single appendix:
%\appendix[Proof of the Zonklar Equations]
% or
%\appendix  % for no appendix heading
% do not use \section anymore after \appendix, only \section*
% is possibly needed

% use appendices with more than one appendix
% then use \section to start each appendix
% you must declare a \section before using any
% \subsection or using \label (\appendices by itself
% starts a section numbered zero.)
%

%\appendices
%\section{Proof of the First Zonklar Equation}
%Appendix one text goes here.
%
%% you can choose not to have a title for an appendix
%% if you want by leaving the argument blank
%\section{}
%Appendix two text goes here.

% use section* for acknowledgment
% \section*{Acknowledgment}
% The authors would like to thank Prof. Dongbin Zhao for his support to this work.

% Can use something like this to put references on a page
% by themselves when using endfloat and the captionsoff option.
\ifCLASSOPTIONcaptionsoff
  \newpage
\fi

% trigger a \newpage just before the given reference
% number - used to balance the columns on the last page
% adjust value as needed - may need to be readjusted if
% the document is modified later
%\IEEEtriggeratref{8}
% The "triggered" command can be changed if desired:
%\IEEEtriggercmd{\enlargethispage{-5in}}

% references section

% can use a bibliography generated by BibTeX as a .bbl file
% BibTeX documentation can be easily obtained at:
% http://mirror.ctan.org/biblio/bibtex/contrib/doc/
% The IEEEtran BibTeX style support page is at:
% http://www.michaelshell.org/tex/ieeetran/bibtex/
\bibliographystyle{IEEEtran}
% argument is your BibTeX string definitions and bibliography database(s)
\bibliography{IEEEabrv,tnnls_relation_gait}
%
% <OR> manually copy in the resultant .bbl file
% set second argument of \begin to the number of references
% (used to reserve space for the reference number labels box)
%\begin{thebibliography}{1}
%\bibitem{IEEEhowto:kopka}
%H.~Kopka and P.~W. Daly, \emph{A Guide to \LaTeX}, 3rd~ed.\hskip 1em plus
%  0.5em minus 0.4em\relax Harlow, England: Addison-Wesley, 1999.
%\end{thebibliography}

% biography section
%
% If you have an EPS/PDF photo (graphicx package needed) extra braces are
% needed around the contents of the optional argument to biography to prevent
% the LaTeX parser from getting confused when it sees the complicated
% \includegraphics command within an optional argument. (You could create
% your own custom macro containing the \includegraphics command to make things
% simpler here.)
%\begin{IEEEbiography}[{\includegraphics[width=1in,height=1.25in,clip,keepaspectratio]{mshell}}]{Michael Shell}
% or if you just want to reserve a space for a photo:

%\begin{IEEEbiography}{Michael Shell}
%Biography text here.
%\end{IEEEbiography}
%
%% if you will not have a photo at all:
%\begin{IEEEbiographynophoto}{John Doe}
%Biography text here.
%\end{IEEEbiographynophoto}

% insert where needed to balance the two columns on the last page with
% biographies
% \newpage

%\begin{IEEEbiographynophoto}{Jane Doe}
%Biography text here.
%\end{IEEEbiographynophoto}

%\begin{IEEEbiography}[{\includegraphics[width=1in,height=1.25in,clip,keepaspectratio]{photos/hsh.pdf}}]{Saihui Hou}
%% \begin{IEEEbiographynophoto}{Saihui Hou}
%	received the B.E. and Ph.D. degrees from University of Science and Technology of China in 2014 and 2019, respectively.
%    %
%    He is currently an Assistant Professor with School of Artificial Intelligence, Beijing Normal University.
%    %
%    His research interests include computer vision and machine learning.
%    %
%    He focuses on gait recognition which aims to identify different people according to the walking patterns.
%% \end{IEEEbiographynophoto}
%\end{IEEEbiography}
%
%\begin{IEEEbiography}[{\includegraphics[width=1in,height=1.25in,clip,keepaspectratio]{photos/lx.pdf}}]{Xu Liu}
%% \begin{IEEEbiographynophoto}{Xu Liu}
%	received the B.E. and Ph.D. degrees from University of Science and Technology of China in 2013 and 2018, respectively.
%    %
%    He is currently a Research Scientist with Watrix Technology Limited Co. Ltd.
%    %
%    His research interests include gait recognition, object detection and image segmentation.
%% \end{IEEEbiographynophoto}
%\end{IEEEbiography}
%
%\begin{IEEEbiography}[{\includegraphics[width=1in,height=1.25in,clip,keepaspectratio]{photos/ccs.pdf}}]{Chunshui Cao}
%% \begin{IEEEbiographynophoto}{Chunshui Cao}
%	received the B.E. and Ph.D. degrees from University of Science and Technology of China in 2013 and 2018, respectively.
%    %
%    During his Ph.D. study, he joined Center for Research on Intelligent Perception and Computing, National Laboratory of Pattern Recognition, Institute of Automation, Chinese Academy of Sciences.
%    %
%    From 2018 to 2020, he worked as a Postdoctoral Fellow with PBC School of Finance, Tsinghua University.
%    %
%    He is currently a Research Scientist with Watrix Technology Limited Co. Ltd.
%    %
%    His research interests include pattern recognition, computer vision and machine learning.
%% \end{IEEEbiographynophoto}
%\end{IEEEbiography}
%
%\begin{IEEEbiography}[{\includegraphics[width=1in,height=1.25in,clip,keepaspectratio]{photos/hyz.pdf}}]{Yongzhen Huang}
%% \begin{IEEEbiographynophoto}{Yongzhen Huang}
%	received the B.E. degree from Huazhong University of Science and Technology in 2006, and the Ph.D. degree from Institute of Automation, Chinese Academy of Sciences in 2011.
%    %
%    He is currently an Associate Professor with School of Artificial Intelligence, Beijing Normal University.
%    %
%    He has published one book and more than 80 papers at international journals and conferences such as TPAMI, IJCV, TIP, TSMCB, TMM, TCSVT, CVPR, ICCV, ECCV, NIPS, AAAI.
%    %
%    His research interests include pattern recognition, computer vision and machine learning.
%% \end{IEEEbiographynophoto}
%\end{IEEEbiography}


\end{document}

\fi
