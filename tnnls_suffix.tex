% !TEX root = tnnls_relation_gait.tex

% if have a single appendix:
%\appendix[Proof of the Zonklar Equations]
% or
%\appendix  % for no appendix heading
% do not use \section anymore after \appendix, only \section*
% is possibly needed

% use appendices with more than one appendix
% then use \section to start each appendix
% you must declare a \section before using any
% \subsection or using \label (\appendices by itself
% starts a section numbered zero.)
%

%\appendices
%\section{Proof of the First Zonklar Equation}
%Appendix one text goes here.
%
%% you can choose not to have a title for an appendix
%% if you want by leaving the argument blank
%\section{}
%Appendix two text goes here.

% use section* for acknowledgment
% \section*{Acknowledgment}
% The authors would like to thank Prof. Dongbin Zhao for his support to this work.

% Can use something like this to put references on a page
% by themselves when using endfloat and the captionsoff option.
\ifCLASSOPTIONcaptionsoff
  \newpage
\fi

% trigger a \newpage just before the given reference
% number - used to balance the columns on the last page
% adjust value as needed - may need to be readjusted if
% the document is modified later
%\IEEEtriggeratref{8}
% The "triggered" command can be changed if desired:
%\IEEEtriggercmd{\enlargethispage{-5in}}

% references section

% can use a bibliography generated by BibTeX as a .bbl file
% BibTeX documentation can be easily obtained at:
% http://mirror.ctan.org/biblio/bibtex/contrib/doc/
% The IEEEtran BibTeX style support page is at:
% http://www.michaelshell.org/tex/ieeetran/bibtex/
\bibliographystyle{IEEEtran}
% argument is your BibTeX string definitions and bibliography database(s)
\bibliography{IEEEabrv,tnnls_relation_gait}
%
% <OR> manually copy in the resultant .bbl file
% set second argument of \begin to the number of references
% (used to reserve space for the reference number labels box)
%\begin{thebibliography}{1}
%\bibitem{IEEEhowto:kopka}
%H.~Kopka and P.~W. Daly, \emph{A Guide to \LaTeX}, 3rd~ed.\hskip 1em plus
%  0.5em minus 0.4em\relax Harlow, England: Addison-Wesley, 1999.
%\end{thebibliography}

% biography section
%
% If you have an EPS/PDF photo (graphicx package needed) extra braces are
% needed around the contents of the optional argument to biography to prevent
% the LaTeX parser from getting confused when it sees the complicated
% \includegraphics command within an optional argument. (You could create
% your own custom macro containing the \includegraphics command to make things
% simpler here.)
%\begin{IEEEbiography}[{\includegraphics[width=1in,height=1.25in,clip,keepaspectratio]{mshell}}]{Michael Shell}
% or if you just want to reserve a space for a photo:

%\begin{IEEEbiography}{Michael Shell}
%Biography text here.
%\end{IEEEbiography}
%
%% if you will not have a photo at all:
%\begin{IEEEbiographynophoto}{John Doe}
%Biography text here.
%\end{IEEEbiographynophoto}

% insert where needed to balance the two columns on the last page with
% biographies
% \newpage

%\begin{IEEEbiographynophoto}{Jane Doe}
%Biography text here.
%\end{IEEEbiographynophoto}

%\begin{IEEEbiography}[{\includegraphics[width=1in,height=1.25in,clip,keepaspectratio]{photos/hsh.pdf}}]{Saihui Hou}
%% \begin{IEEEbiographynophoto}{Saihui Hou}
%	received the B.E. and Ph.D. degrees from University of Science and Technology of China in 2014 and 2019, respectively.
%    %
%    He is currently an Assistant Professor with School of Artificial Intelligence, Beijing Normal University.
%    %
%    His research interests include computer vision and machine learning.
%    %
%    He focuses on gait recognition which aims to identify different people according to the walking patterns.
%% \end{IEEEbiographynophoto}
%\end{IEEEbiography}
%
%\begin{IEEEbiography}[{\includegraphics[width=1in,height=1.25in,clip,keepaspectratio]{photos/lx.pdf}}]{Xu Liu}
%% \begin{IEEEbiographynophoto}{Xu Liu}
%	received the B.E. and Ph.D. degrees from University of Science and Technology of China in 2013 and 2018, respectively.
%    %
%    He is currently a Research Scientist with Watrix Technology Limited Co. Ltd.
%    %
%    His research interests include gait recognition, object detection and image segmentation.
%% \end{IEEEbiographynophoto}
%\end{IEEEbiography}
%
%\begin{IEEEbiography}[{\includegraphics[width=1in,height=1.25in,clip,keepaspectratio]{photos/ccs.pdf}}]{Chunshui Cao}
%% \begin{IEEEbiographynophoto}{Chunshui Cao}
%	received the B.E. and Ph.D. degrees from University of Science and Technology of China in 2013 and 2018, respectively.
%    %
%    During his Ph.D. study, he joined Center for Research on Intelligent Perception and Computing, National Laboratory of Pattern Recognition, Institute of Automation, Chinese Academy of Sciences.
%    %
%    From 2018 to 2020, he worked as a Postdoctoral Fellow with PBC School of Finance, Tsinghua University.
%    %
%    He is currently a Research Scientist with Watrix Technology Limited Co. Ltd.
%    %
%    His research interests include pattern recognition, computer vision and machine learning.
%% \end{IEEEbiographynophoto}
%\end{IEEEbiography}
%
%\begin{IEEEbiography}[{\includegraphics[width=1in,height=1.25in,clip,keepaspectratio]{photos/hyz.pdf}}]{Yongzhen Huang}
%% \begin{IEEEbiographynophoto}{Yongzhen Huang}
%	received the B.E. degree from Huazhong University of Science and Technology in 2006, and the Ph.D. degree from Institute of Automation, Chinese Academy of Sciences in 2011.
%    %
%    He is currently an Associate Professor with School of Artificial Intelligence, Beijing Normal University.
%    %
%    He has published one book and more than 80 papers at international journals and conferences such as TPAMI, IJCV, TIP, TSMCB, TMM, TCSVT, CVPR, ICCV, ECCV, NIPS, AAAI.
%    %
%    His research interests include pattern recognition, computer vision and machine learning.
%% \end{IEEEbiographynophoto}
%\end{IEEEbiography}


\end{document}
